To summarise our results, according to A. E. Vinogradov, we saw that increased GC percentages in genomes resulted in slightly less thermostability, but with increased DNA's bendability, which could hint us that places with higher GC percentages are more likely to be coding regions.

When comparing GC counts in different yeast species from different databases, we saw a similarity of GC percentages across all yeast species, which was expected. The differences in outliers between same species of yeast could be explained by having different data from different databases and different year that the genome files were collected. For better results, in the future it would be more preferred to use genome files from a similar timeframe.
