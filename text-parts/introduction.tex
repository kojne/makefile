This report outlines the methods and findings of a project conducted as part of our GNU/Linux Type Operating Systems project, focusing on the analysis of GC content in yeast genomes. The GC content, representing the proportion of guanine and cytosine bases in DNA, is an important genetic parameter that can provide insights into genome organization and function.

Our approach involved utilizing Bash scripting, a tool used in data manipulation and analysis, to handle and process genome data. We specifically targeted yeast genomes for our study, given their significance in both biological research and industry. The main objective was to examine the variations in GC content across different yeast species, provide aquired data via boxplots and find insights and use this gathered data to compare with scientific literature.

According to A. E. Vinogradov's published paper "DNA helix: the importance of being GC-rich", it was assumed that higher counts of GC correlated with higher DNA thermostability and bendability, thus different ectothermic and exothermic organisms \cite{Vinogradov2003}.



